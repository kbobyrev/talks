\documentclass{beamer}

\mode<presentation> {
\usetheme{CambridgeUS}
}

\usepackage{algorithmic}
\usepackage{hyperref}

%-------------------------------------------------------------------------------
% Presentation meta
%-------------------------------------------------------------------------------

\title{Multi-step RL: Unifying Algorithm}

\author{Kirill Bobyrev}

\date{\today}

\begin{document}

\begin{frame}
\titlepage
\end{frame}

\begin{frame}
\frametitle{Plan}
\tableofcontents
\end{frame}

%-------------------------------------------------------------------------------
% Slides
%-------------------------------------------------------------------------------

%-------------------------------------------------------------------------------
\section{Introduction}
%-------------------------------------------------------------------------------

\begin{frame}
  \frametitle{Results}
\end{frame}

%-------------------------------------------------------------------------------
\section{From MC and one-step TD to multi-step Bootstrapping}
%-------------------------------------------------------------------------------

\begin{frame}
  \frametitle{Monte Carlo methods}
\end{frame}

\begin{frame}
  \frametitle{One-step TD methods}
\end{frame}

\begin{frame}
  \frametitle{$n$-step TD methods}
\end{frame}

%-------------------------------------------------------------------------------
\section{Algorithm}
%-------------------------------------------------------------------------------

\begin{frame}
  \frametitle{$Q(\sigma)$ algorithm}
  \begin{algorithmic}
    \STATE Initialize $S_0 \neq {terminal}$
    \STATE Select $A_0$ according to $\pi(. | S_0)$
    \STATE Store $S_0$, $A_0$, $Q(S_0, A_0)$
    \FOR{$t = 0, \ldots, T + n - 1$}
      \IF{$t < T$}
        \STATE Take Action $A_t$, observe $R$ and store $S_{t + 1}$
      \ENDIF
    \ENDFOR
  \end{algorithmic}
\end{frame}

\begin{frame}
  \frametitle{Intuition and Examples}
\end{frame}

%-------------------------------------------------------------------------------
\section{Conclusion}
%-------------------------------------------------------------------------------

\begin{frame}
\frametitle{Synopsis}
\begin{itemize}
    \item $Q(\sigma)$ unifies $n$-step Sarsa and Tree-backup
    \item $Q(\sigma)|_{\sigma=0}$ is Tree-backup
    \item $Q(\sigma)|_{\sigma=1}$ is $n$-step Sarsa
\end{itemize}
\end{frame}

\begin{frame}
  \frametitle<presentation>{References}

  \begin{thebibliography}{10}

  \beamertemplatebookbibitems

  \beamertemplatearticlebibitems

  \bibitem{QSigma}
    Kristopher De Asis, J. Fernando Hernandez-Garcia, G. Zacharias Holland,
    Richard S. Sutton.
    \newblock {\href{https://arxiv.org/abs/1703.01327}{\em Multi-step
      Reinforcement Learning: A Unifying Algorithm}}.
    \newblock arXiv, 3 Mar 2017.

  \bibitem{RLBook} Richard S. Sutton, Andrew G. Barto.  \newblock
    {\href{http://incompleteideas.net/sutton/book/the-book-2nd.html}{\em
      Reinforcement Learning: An Introduction}}.
    \newblock MIT Press, Cambridge, MA, 19 Jun 2017 Draft.

  \end{thebibliography}
\end{frame}

%-------------------------------------------------------------------------------

\begin{frame}
\Huge{\centerline{The End}}
\end{frame}

%-------------------------------------------------------------------------------

\end{document}
